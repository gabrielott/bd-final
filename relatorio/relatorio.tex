\documentclass[12pt]{article}

\usepackage{sbc-template}
\usepackage{graphicx,url}
\usepackage[utf8]{inputenc}
\usepackage[brazil]{babel}
     
\sloppy

\title{Relatório da Implementação do módulo
	de cadastro de questionário }

\author{
	Cristian Vilela\inst{1},
	Filipe Castelo Branco\inst{1},
	Gabriel da Fonseca Ottoboni Pinho\inst{1},\\
	João Pedro Cavalcante\inst{1},
	Rodrigo Delpreti de Siqueira\inst{1}
}


\address{Instituto de Computação --
	Universidade Federal do Rio de Janeiro (UFRJ)\\
}

\begin{document} 

\maketitle

\begin{abstract}
  This report describes a web project implementation
  to be used in form registry, in accordance to the
  VODAN BR support database schemas.
\end{abstract}
     
\begin{resumo} 
  Este relatório descreve a implementação
  de um projeto web
  para cadastro de formulário,
  segundo requisitos da base de dados
  de apoio VODAN BR.
\end{resumo}


\section{Introdução}

No desenvolvimento da implementação
apresentada no presente relatório,
utilizaram-se as frameworks React e Laravel
para o desenvolvimento do front-end e back-end
respectivamente.
Tais escolhas foram baseadas no conhecimento prévio
da equipe, bem como no interesse de aprendizado dos membros.

\section{Interface} \label{sec:firstpage}

A interface da aplicação visa a montagem
dos questionários à partir de botões e dropdowns,
estruturados conforme a hierarquia designada na especificação.
Estes componentes podem ser visualizados nas imagens abaixo.

Após a montagem do questionário,
basta clicar em salvar ao final,
quando será então exibido um aviso de confirmação
e o usuário será retornado à tela inicial.

\section{Back-end}

O backend feito em Laravel reedita as procedures
definidas em SQL

\section{Integração}

Section titles must be in boldface, 13pt, flush left. There should be an extra
12 pt of space before each title. Section numbering is optional. The first
paragraph of each section should not be indented, while the first lines of
subsequent paragraphs should be indented by 1.27 cm.

\subsection{Subsections}

The subsection titles must be in boldface, 12pt, flush left.

\section{Figures and Captions}\label{sec:figs}


Figure and table captions should be centered if less than one line
(Figure~\ref{fig:exampleFig1}), otherwise justified and indented by 0.8cm on
both margins, as shown in Figure~\ref{fig:exampleFig2}. The caption font must
be Helvetica, 10 point, boldface, with 6 points of space before and after each
caption.

In tables, try to avoid the use of colored or shaded backgrounds, and avoid
thick, doubled, or unnecessary framing lines. When reporting empirical data,
do not use more decimal digits than warranted by their precision and
reproducibility. Table caption must be placed before the table (see Table 1)
and the font used must also be Helvetica, 10 point, boldface, with 6 points of
space before and after each caption.

\section{Images}

All images and illustrations should be in black-and-white, or gray tones,
excepting for the papers that will be electronically available (on CD-ROMs,
internet, etc.). The image resolution on paper should be about 600 dpi for
black-and-white images, and 150-300 dpi for grayscale images.  Do not include
images with excessive resolution, as they may take hours to print, without any
visible difference in the result. 

\section{References}

\cite{1}, \cite{2}, \cite{3}.

\bibliographystyle{sbc}
\bibliography{sbc-template}

\end{document}
